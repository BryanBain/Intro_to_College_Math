\documentclass{article}
\usepackage{amsmath, tikz, tcolorbox, array, multicol, sfmath, enumerate, pgfplots}
\renewcommand{\familydefault}{\sfdefault}
\pgfplotsset{compat=newest}
\usetikzlibrary{arrows.meta}
\everymath{\displaystyle}
\tikzset{>=stealth}
\usepackage[top = 0.25in, bottom = 0.25in, left = 1.25in, right = 1.25in]{geometry}
\pagestyle{empty}
\raggedright

\newcounter{example}[section]
\newenvironment{example}[1][]{\refstepcounter{example}\par\medskip
   {\color{red}\textbf{Example~\theexample. #1}}}{\medskip}

\begin{document}

\section*{Derivatives of Constants, Powers, and Sums}

\begin{tcolorbox}[colframe=orange!70!white, coltitle=black, title=\textbf{Summary}]
\begin{enumerate}
    \item 
\end{enumerate}
\end{tcolorbox}
\vspace{1in}

\subsection*{Derivative of a Constant Function}

The derivative of a constant is 0.

\begin{example}
Find $\frac{dy}{dx}$ for each.
\begin{multicols}{2}
\begin{enumerate}[(a)]
    \item $f(x) = 7$
    \item $g(x) = \sqrt[3]{2}$
\end{enumerate}
\end{multicols}
\end{example}

\subsection*{Power Rule}

If $f(x) = x^n$, where $n$ is any real number, then $f'(x)= nx^{n-1}$. \newline\\

For radicals, remember that $\sqrt[r]{x^p} = x^{p/r}$.

\begin{example}
Find $\frac{dy}{dx}$ for each.
\begin{multicols}{2}
\begin{enumerate}[(a)]
    \item $f(x) = x^4$
    \item $g(x) = x^{1.32}$
\end{enumerate}
\end{multicols}
\begin{multicols}{2}
\begin{enumerate}[(a)]  \setcounter{enumi}{2}
    \item $y = \sqrt{x}$
    \item $g(x) = \frac{1}{x^3}$
\end{enumerate}
\end{multicols}
\end{example}

\subsection*{Constant Multiple Rule}

\begin{itemize}
    \item For $f(x) = 2x^3$, $f'(x) = 6x^2$
    \item For $f(x) = 3x^5$, $f'(x) = 15x^4$
    \item For $f(x) = -2x^4$, $f'(x) = -8x^3$
\end{itemize}

For $f(x) = c \cdot x^n$, $f'(x) = n\cdot c \cdot x^{n-1}$

\begin{example}
Differentiate each.
\begin{multicols}{3}
\begin{enumerate}[(a)]
    \item $g(x) = 1.2x^5$
    \item $y = \frac{1}{7x^3}$
    \item $f(x) = \frac{2}{3}\sqrt[5]{x}$
\end{enumerate}
\end{multicols}
\end{example}

\subsection*{Sum and Difference Rules}

To differentiate a sum/difference of 2 (or more functions), \\
differentiate each function separately and add/subtract the results.

\begin{example}
Differentiate each.
\begin{multicols}{3}
\begin{enumerate}[(a)]
    \item $f(x) = 3x^2 + 2x - 1$
    \item $g(x) = \frac{1}{2}x^3 - \frac{3}{2}x^{-1}$
    \item $y = 3x^4 + 2\sqrt{x} - \frac{2}{x^2}$
\end{enumerate}
\end{multicols}
\end{example}

\subsection*{Applications}

\begin{example}
A coconut falls from a tree that is 75 feet tall. \\
Its height above the ground after $t$ seconds is given by
\[
s(t) = 75-16t^2
\]
where $s(t)$ is measured in feet and is the \textbf{position function}.
\begin{enumerate}[(a)]
    \item The \textbf{velocity function}, $v(t)$, is $s'(t)$. Determine $v(t)$.
    \item Compute and interpret $s(2)$ and $v(2)$.
    \item When does the coconut hit the ground?
\end{enumerate}
\end{example}

\begin{example}
A refrigerator company determines the total cost function for producing fridges is
\[
C(x) = 2x^2 + 15x + 1500
\]
where $x$ is the weekly production of fridges and $C(x)$ is the total cost (in dollars). \\

The revenue function is given as
\[
R(x) = -0.3x^2 + 460x
\]
where $R(x)$ is in dollars.
\begin{enumerate}[(a)]
    \item Determine $P(x)$ and $P'(x)$.
    \item Evaluate and interpret $P(20)$ and $P'(20)$.
\end{enumerate}
\end{example}

\end{document}
