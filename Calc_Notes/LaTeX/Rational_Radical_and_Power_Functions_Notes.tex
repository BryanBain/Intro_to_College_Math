\documentclass{article}
\usepackage{amsmath, tikz, tcolorbox, array, sfmath, enumerate, pgfplots, marvosym, amssymb, multicol}
\renewcommand{\familydefault}{\sfdefault}
\pgfplotsset{compat=newest}
\usepgfplotslibrary{fillbetween}
% \everymath{\displaystyle}
\tikzset{>=stealth}
\usepackage[top = 0.25in, bottom = 0.25in, left = 1.25in, right = 1.25in]{geometry}
\pagestyle{empty}
\raggedright

\tikzset{>=stealth}
\tikzstyle{input} = [circle, text centered, radius = 1cm, draw = black]
\tikzstyle{function} = [rectangle, text centered, minimum width = 2cm, minimum height = 1cm, draw = black]

\newcounter{example}[section]
\newenvironment{example}[1][]{\refstepcounter{example}\par\medskip
   {\color{red}\textbf{Example~\theexample. #1}}}{\medskip}

\begin{document}

\section*{Rational, Radical, and Power Functions}

\begin{tcolorbox}[colframe=orange!70!white, coltitle=black, title=\textbf{Summary}]
\begin{enumerate}
    \item The denominator of a rational function can never equal 0.
    \item Each $x$-value that causes this is either a vertical asymptote or a hole in the graph.
    \item You can't take even roots of negative numbers.
\end{enumerate}
\end{tcolorbox}
\vspace{1in}

\subsection*{Rational Functions}

Rational functions have the form $y = \frac{f(x)}{g(x)}$ where $g(x) \neq 0$.
\vspace{0.5in}

\begin{example}
Given the rational function $f(x) = \frac{2x}{x-3}$
\begin{enumerate}[(a)]
    \item Determine the domain. \vfill 
    \item Complete the tables below and describe what happens as $x$ approaches 3.    \newline\\
    
    \begin{minipage}{0.35\textwidth}
    \begin{tabular}{c|c}    \setlength{\extrarowheight}{6pt}
    $\pmb{x}$ & $\pmb{f(x)}$ \\ \hline 
    2 & \\[6pt] \hline
    2.9 & \\[6pt]   \hline
    2.99 & \\[6pt]  \hline
    2.999 & \\[6pt] \hline
    2.9999 & \\[6pt]
    \end{tabular}
    \end{minipage}
    \begin{minipage}{0.25\textwidth}
    \begin{tabular}{c|c}    \setlength{\extrarowheight}{6pt}
    $\pmb{x}$ & $\pmb{f(x)}$ \\ \hline 
    4 & \\[6pt] \hline
    3.1 & \\[6pt]   \hline
    3.01 & \\[6pt]  \hline
    3.001 & \\[6pt] \hline
    3.0001 & \\[6pt]
    \end{tabular}
    \end{minipage}
\end{enumerate}
\end{example}

\vfill 
\newpage 

Notice how the output values get out of control: 
\[
f(x) \to -\infty \quad \text{or} \quad f(x) \to \infty
\]

Since $x$ can't equal 3, the graph of $\frac{2x}{x-3}$ will never cross the vertical line $x = 3$. \newline\\

This line is called a {\color{blue}\textbf{vertical asymptote}}. 
\vspace{1in}

At each of the values that cause the denominator to = 0, \\ there will be \textbf{only one of two things there}: 
\begin{itemize}
    \item A vertical asymptote
    \item Or a hole in the graph
\end{itemize}

\vspace{1in}

\textsc{How to Tell Your Asymptote From a Hole In the Graph:}
    \begin{enumerate}
        \item {\color{red}\textbf{Factor}} both numerator and denominator {\color{red}\textbf{**completely**}} and then {\color{blue}\textbf{simplify that}}. 
        \item For {\color{violet}\textbf{each domain issue}} in the \underline{denominator}, evaluate the \textbf{simplified function} at those $x$-values:
        \begin{itemize}
            \item If you still get a 0 in the denominator (typically an error message) you have a \textbf{vertical asymptote}.  
            \item Otherwise, there is a hole in the graph there ($y$-coordinate is that output value)    
        \end{itemize}
    \end{enumerate}
    
\vspace{1in}

\begin{example}
Determine the equations for any vertical asymptote(s) and/or exact coordinates of any holes in the graph.
\begin{multicols}{2}
\begin{enumerate}[(a)]
    \item $\frac{x^2-4}{x^2-x-2}$
    \item $\frac{x+5}{x^2+8x+15}$
\end{enumerate}
\end{multicols}
\end{example}

\vfill 
\newpage 

\subsection*{Radical Functions}

\begin{itemize}
    \item \emph{Can't} take square roots (or even roots in general) of negative numbers: $\sqrt{\geq 0}$
    \item \emph{Can} take cube roots (or odd roots in general) of any number.
\end{itemize}
\vspace{1in}

\begin{example}
A sapling of a type of tree grows according to the equation
\[
g(x) = 10\sqrt{x} + 0.75
\]
where $x$ is the number of years since planting and $g(x)$ is the height of the tree.
\begin{enumerate}[(a)]
    \item How tall will it be on its 20th birthday? \vspace{2in}
    \item How long will it take to be 30 feet tall?
\end{enumerate}
\end{example}
\vfill 
\newpage 

\subsection*{Power Functions}

\begin{itemize}
    \item Has form $f(x) = a \cdot x^b$
    \item If $a$ is positive:
    \begin{itemize}
        \item Opens downward if $0 < b < 1$
        \item Opens upward if $b > 1$
    \end{itemize}
\end{itemize}

\vspace{1.5in}

\begin{example}
The amount of money spent (in billions) on R\&D each year is given by 
\[
f(x) = 19.66x^{0.74} \quad 5 \leq x \leq 21
\]
where $x$ is the number of years since 2000.
\begin{enumerate}[(a)]
    \item Evaluate and interpret $f(12)$ \vspace{2in}
    \item Find the average rate of change for the amount spent between 2008 and 2012.
\end{enumerate}
\end{example}

\end{document}
