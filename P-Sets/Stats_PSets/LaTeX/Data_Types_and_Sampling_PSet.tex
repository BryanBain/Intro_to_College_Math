\documentclass{article}
\usepackage{amsmath, sfmath, multicol}
\usepackage[margin = 0.5in]{geometry}
\renewcommand{\familydefault}{\sfdefault}
\setlength\parindent{0cm}
\pagestyle{empty}

\newcounter{pset}
\newcounter{key}


\begin{document}

\subsection*{Data Types and Sampling P-Set}

\subsubsection*{Samples and Populations}

For questions 1--3, identify the sample and population. Then determine if the sample is likely to be representative of the population.
\begin{enumerate}
    \item An employee at a local ice cream parlor asks three customers if they like chocolate ice cream.
    \item 100,000 randomly selected adults were asked whether they drink at least 48 oz. of water each day and 45\% said yes.
    \item In a poll of 50,000 randomly selected high school juniors and seniors, when asked if they have a car, 53\% said yes.
\end{enumerate} \setcounter{pset}{\value{enumi}}

\subsubsection*{Statistical and Practical Significance}

\begin{enumerate} \setcounter{enumi}{\value{pset}}
    \item On an exam with 201 true-false questions, Charlie answered 108 of them correctly. The chances of Charlie obtaining this score via guessing is about 1 in 7. Are these results statistically significant? Why or why not?
    \item A coach uses a new technique in training middle distance runners. The times, in seconds, for 8 different athletes to run 800 meters before and after this training are shown:
    \begin{center}
        \begin{tabular}{c|c|c|c|c|c|c|c|c}
            \textbf{Athlete} & A & B & C & D & E & F & G & H \\ \hline 
            \textbf{Before} & 115.2 & 114 & 116.4 & 119.8 & 110.9 & 112.4 & 111.5 & 117.3 \\ \hline 
            \textbf{After} & 112.9 & 112.7 & 114 & 120.6 & 109.1 & 109.1 & 107.9 & 113.4
        \end{tabular}
    \end{center}
    \begin{enumerate}
        \item Does the conclusion that the technique is effective appear to be supported with statistical significance? Why or why not?
        \item Does the conclusion that the technique is effective appear to have practical significance? Why or why not?
    \end{enumerate}
    \item A researcher investigated whether following a vegetarian diet could help to reduce blood pressure. For a sample of 85 people who followed a vegetarian diet, the mean systolic blood pressure was 124 torr and the sample of 75 people who followed a non-vegetarian diet was 138 torr. \newline\\
    
    Statistical methods show if a vegetarian diet had no effect on blood pressure, there would be less than 1 chance in 100 of getting these results.
    \begin{enumerate}
        \item Does the result have statistical significance? Why or why not?
        \item Does the result have practical significance? Why or why not?
    \end{enumerate}
\end{enumerate}     \setcounter{pset}{\value{enumi}}

\subsubsection*{Qualitative and Quantitative Data}

Determine if each of the following represents qualitative or quantitative data.
\begin{enumerate} \setcounter{enumi}{\value{pset}}
    \item The colors of book covers on a bookshelf
    \item The number of calls received at a company's help desk
    \item How many hours I spend each day eating
\end{enumerate}     \setcounter{pset}{\value{enumi}}

\subsubsection*{Discrete and Continuous Data}

Determine whether each quantitative variable is discrete or continuous.
\begin{enumerate} \setcounter{enumi}{\value{pset}}
    \item The temperature of a cup of coffee.
    \item The number of times a dog wants to go outside.
    \item The number of times I have to lie to my boss about something in a given week.
    \item The speed at which I drive when someone is tailing me.
\end{enumerate}     \setcounter{pset}{\value{enumi}}

\subsubsection*{Observational Studies and Experiments}

Classify each as either an observational study or an experiment.
\begin{enumerate} \setcounter{enumi}{\value{pset}}
    \item A marketing firm conducts a survey to find out how many people are going to sue them. Of the 100 surveyed, 15 said they would.
    \item A clinic gives a drug to a group of 100 patients and a placebo to another group of 100 patients to find out if the drug has any effect on the patients' illness.
    \item A quality control specialist compares the output from a machine with a new part to the outputs of machines without the new part.
    \item A stock analyst compares the relationship between stock prices and earnings per share to help select a stock for investment.
\end{enumerate}     \setcounter{pset}{\value{enumi}}

\subsubsection*{Sampling Methods}

Classify each by the sampling method used.
\begin{enumerate} \setcounter{enumi}{\value{pset}}
    \item A market researcher selects 500 drives under the age of 30 and 500 drives 30 years and older.
    \item A pollster uses a computer to generate 500 random numbers, then interviews the voters corresponding to those numbers.
    \item To avoid working late, a quality control analyst inspects the first 100 items produced in a day.
    \item An education researcher randomly selects 58 high schools and interviews all of the teachers at each school.
    \item A sample consists of every 49th student at a large school.
    \item A tax auditor selects every 1000\textsuperscript{th} income tax return that is received.
\end{enumerate}     \setcounter{pset}{\value{enumi}}

\newpage 

\texttt{Key}
\begin{enumerate}
    \item Sample: 3 customers; Population: all ice cream customers; Not likely representative
    \item Sample: 100,000 selected adults; Population: all adults; Likely representative
    \item Sample: 50,000 selected juniors and seniors; Population: all juniors and seniors; Likely representative
    
    \item Not statistically significant; if he guessed, Charlie would get about 100 questions right and 108 is not that much different from 100.
    \item 
    \begin{enumerate}
        \item Yes. Almost all runners have considerably faster times after the training.
        \item Yes. The differences appear to be substantial enough to implement training of more athletes.
    \end{enumerate}
    \item 
    \begin{enumerate}
        \item Yes. The group following a vegetarian diet had a substantially lower blood pressure. If the vegetarian diet did not help to reduce blood pressure, there would be a very small chance of getting those results.
        \item Yes. The difference in blood pressure appears substantial enough to help reduce blood pressure.
    \end{enumerate}
    
    \item Qualitative
    \item Quantitative
    \item Quantitative
    
    \item Continuous
    \item Discrete
    \item Discrete
    \item Continuous
    
    \item Observational Study
    \item Experiment
    \item Experiment
    \item Observational Study
    
    \item Stratified
    \item Random
    \item Convenience
    \item Cluster
    \item Systematic
    \item Systematic
\end{enumerate}


\end{document}
