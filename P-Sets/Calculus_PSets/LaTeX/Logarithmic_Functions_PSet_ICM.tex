\documentclass{article}
\usepackage{amsmath, sfmath, multicol}
\renewcommand{\familydefault}{\sfdefault}
\raggedright
\pagestyle{empty}
\usepackage[margin = 0.5in]{geometry}

\newcounter{pset}
\newcounter{key}

\begin{document}

\subsection*{Logarithmic Functions P-Set} 

Re-write each as a logarithmic expression.
\begin{multicols}{4}
\begin{enumerate}
    \item $2^5 = 32$
    \item $3^{-4} = \frac{1}{81}$
    \item $81^{3/4} = 27$
    \item $e^0 = 1$
\end{enumerate}     \setcounter{pset}{\value{enumi}}
\end{multicols}

Use the properties of logarithms to rewrite each of the following.
\begin{multicols}{3}
\begin{enumerate}   \setcounter{enumi}{\value{pset}}
    \item $\log_2\left(\frac{3}{5}\right)$
    \item $\log(8 \cdot 20)$
    \item $\ln\left(\sqrt{26}\right)$
\end{enumerate}     \setcounter{pset}{\value{enumi}}
\end{multicols}
\begin{multicols}{3}
\begin{enumerate}   \setcounter{enumi}{\value{pset}}
    \item $\log_6\left(\frac{4}{7}\right)$
    \item $\log(17 \cdot 11)$
    \item $\ln\left(\sqrt[3]{11}\right)$
\end{enumerate}     \setcounter{pset}{\value{enumi}}
\end{multicols}

The number of bacteria in a petri dish can be modeled by the formula $A = A_0e^{kt}$, where
\begin{itemize}
    \item $A$ is the number of bacteria present at any moment in time
    \item $A_0$ is the number of bacteria at the start ($t=0$)
    \item $k$ is a constant
    \item $t$ is the time (in minutes)
\end{itemize}

\begin{enumerate}   \setcounter{enumi}{\value{pset}}
    \item Suppose there are 10,000 bacteria initially present and after 3 minutes, the bacteria count is now 25,000. Find the value of $k$. Round to 3 decimal places.
    \item Using your answer for $k$ above, how long will it take a colony of 10,000 bacteria to grow to a size of 100,000?
    \item How long until the bacteria population reaches 1,000,000?
\end{enumerate}     \setcounter{pset}{\value{enumi}}
\vfill 

\dotfill \newline 
\texttt{Key} 

\begin{multicols}{4}
\begin{enumerate}
    \item $\log_2(32) = 5$
    \item $\log_3\left(\frac{1}{81}\right)=-4$
    \item $\log_{81}(27) = \frac{3}{4}$
    \item $\ln(1) = 0$
\end{enumerate}     \setcounter{key}{\value{enumi}}
\end{multicols}
\begin{multicols}{3}
\begin{enumerate}   \setcounter{enumi}{\value{key}}
    \item $\log_2(3) - \log_2(5)$
    \item $\log(8) + \log(20)$
    \item $\frac{1}{2}\ln(26)$
\end{enumerate}     \setcounter{key}{\value{enumi}}
\end{multicols}
\begin{multicols}{3}
\begin{enumerate}   \setcounter{enumi}{\value{key}}
    \item $\log_6(4) - \log_6(7)$
    \item $\log(17) + \log(11)$
    \item $\frac{1}{3}\ln(11)$
\end{enumerate}     \setcounter{key}{\value{enumi}}
\end{multicols}
\begin{multicols}{3}
\begin{enumerate}   \setcounter{enumi}{\value{key}}
    \item $k \approx 0.305$
    \item About 7.55 minutes
    \item About 15.10 minutes
\end{enumerate}     \setcounter{key}{\value{enumi}}
\end{multicols}




\end{document}