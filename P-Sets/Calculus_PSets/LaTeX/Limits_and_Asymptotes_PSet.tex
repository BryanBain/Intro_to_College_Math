\documentclass{article}
\usepackage{amsmath, enumerate, sfmath, multicol, pgfplots}
\pgfplotsset{compat = newest}
\renewcommand{\familydefault}{\sfdefault}
\pagestyle{empty}
\raggedright
\everymath{\displaystyle}
\usepackage[margin = 0.5in]{geometry}
\newcounter{pset}
\begin{document}

\subsection*{Limits and Asymptotes P-Set}

Estimate the following limits. Use the symbols $-\infty$ and $\infty$ where applicable.
\begin{enumerate}
    \item $f(x) = \frac{1}{x^3}$    \quad $\lim_{x \to 0^-}f(x)$
    \begin{center}
    \begin{tabular}{c|c|c|c|c|c}
        $\pmb{x}$ & $-0.1$ & $-0.01$ & $-0.001$ & $-0.0001$ & 0  \\ \hline 
        $\pmb{f(x)}$ & & & & & \\
    \end{tabular}
    \end{center}
    \item $f(x) = \frac{1}{(x-1)^2}$ \quad $\lim_{x \to 1^+} f(x)$
    \begin{center}
    \begin{tabular}{c|c|c|c|c|c}
        $\pmb{x}$ & 1 & 1.0001 & 1.001 & 1.01 & 1.1 \\ \hline 
        $\pmb{f(x)}$ & & & & &  \\
    \end{tabular}
    \end{center}
    \item $f(x) = \frac{x+2}{x^2-x-6}$ \quad $\lim_{x \to 3} f(x)$
    \begin{center}
    \begin{tabular}{c|c|c|c|c|c|c|c|c|c}
        $\pmb{x}$ & $2.9$ & $2.99$ & $2.999$ & $2.9999$ & 3 & 3.0001 & 3.001 & 3.01 & 3.1 \\ \hline 
        $\pmb{f(x)}$ & & & & & & & & & \\
    \end{tabular}
    \end{center}
\end{enumerate} \setcounter{pset}{\value{enumi}}
\bigskip 

State the domain of each rational function. Then find the limit \emph{algebraically}.
\begin{multicols}{4}
\begin{enumerate}   \setcounter{enumi}{\value{pset}}
    \item $\lim_{x \to -2} \frac{x+2}{x^2-x-6}$
    \item $\lim_{x \to 3} \frac{x+2}{x^2-x-6}$
    \item $\lim_{x \to 3} \frac{x-3}{x^2-9}$
    \item $\lim_{x \to 1} \frac{x-1}{x^2-1}$
\end{enumerate} \setcounter{pset}{\value{enumi}}
\end{multicols}
\bigskip

The cost $C(x)$ in thousands of dollars of removing $x\%$ of a city's pollution discharged into a lake is given by
\[
C(x) = \frac{223x}{100-x}
\]
\begin{enumerate}   \setcounter{enumi}{\value{pset}}
    \item Determine a reasonable domain for $C(x)$
    \item Evaluate and interpret $C(40)$
    \item Determine and interpret $\lim_{x \to 100^-} C(x)$
\end{enumerate} \setcounter{pset}{\value{enumi}}
\bigskip 

Pharmacological studies have determined that the amount of medication present in the body is a function of the amount given and how much time has elapsed since the medication was administered. For a certain medication, the amount present $A(t)$, in mL, can be approximated by the function
\[
A(t) = 3.5e^{-0.3t} \quad t \geq 0
\]
where $t$ is the number of hours since the medication was administered.
\begin{enumerate}   \setcounter{enumi}{\value{pset}}
    \item Determine and interpret $A(0)$
    \item Determine and interpret $\lim_{t \to \infty} A(t)$
\end{enumerate} \setcounter{pset}{\value{enumi}}
\bigskip 

Evaluate each of the following.
\begin{multicols}{3}
\begin{enumerate}   \setcounter{enumi}{\value{pset}}
    \item $\lim_{x \to -\infty} \frac{2x+5}{x-1}$
    \item $\lim_{x \to \infty} \frac{3x^2-x+2}{2x^2+x-5}$
    \item $\lim_{x \to -\infty} \frac{3x^2-2x+5}{2x^3+x^2-2x+3}$
\end{enumerate} \setcounter{pset}{\value{enumi}}
\end{multicols}
\bigskip 

The average cost of producing $x$ units of a product is given by 
    \[
    AC(x) = \frac{15,325 + 7.11x}{x}
    \]
\begin{enumerate}   \setcounter{enumi}{\value{pset}}
    \item Determine and interpret $\lim_{x \to \infty} AC(x)$
\end{enumerate} \setcounter{pset}{\value{enumi}}

\newpage 

\texttt{Key}:
\begin{enumerate}
    \item $-\infty$
    \item $\infty$
    \item Does not exist
    \item Domain: $x \neq -2, 3$; $-\tfrac{1}{5}$
    \item Domain: $x \neq -2, 3$; Does not exist
    \item Domain: $x \neq -3, 3$; $\tfrac{1}{6}$
    \item Domain: $x \neq -1, 1$; $\tfrac{1}{2}$
    \item $0 \leq x < 100$
    \item $C(40)\approx 148.7$; removing 40\% of the lake's pollution will cost about \$148,700
    \item $\lim_{x \to 100^-}C(x) = \infty$; The costs increase indefinitely as the percent of pollution removed increases towards 100\%
    \item $A(0) = 3.5$; 3.5 mL of medication were administered
    \item $\lim_{t \to \infty} A(t) = 0$; as time progresses, the amount of medication in the system approaches 0 mL
    \item 2
    \item $\tfrac{3}{2}$
    \item 0
    \item $\lim_{x \to \infty} AC(x) = 7.11$; as the number of units produced increases, the average cost per unit approaches \$7.11
\end{enumerate}



\end{document}
