\documentclass{article}
\usepackage{amsmath, sfmath, multicol}
\renewcommand{\familydefault}{\sfdefault}
\raggedright
\pagestyle{empty}
\usepackage[margin = 0.5in]{geometry}

\newcounter{pset}

\begin{document}

\subsection*{Exponential Functions P-Set} 

Given $f(x) = 6(1.2)^x$, evaluate each. Round to 3 decimal places.
\begin{multicols}{3}
\begin{enumerate}
    \item $f(2)$
    \item $f(-1.1)$
    \item $f(5.8)$
\end{enumerate}     \setcounter{pset}{\value{enumi}}
\end{multicols}

The healing of a 200-square-centimeter wound depends on the amount of time since the wound occurred and can be modeled by
\[
f(x) = 200\left(\frac{4}{5}\right)^x \quad x \geq 0
\]
where $x$ represents the number of days since the wound occurred and $f(x)$ represents the size of the wound in square centimeters.
\begin{enumerate}   \setcounter{enumi}{\value{pset}}
    \item Evaluate and interpret $f(7)$
    \item How many days will it take for the wound to be half its original size?
    \item Find and interpret the average rate of change of the function in the interval [1, 7].
\end{enumerate} \setcounter{pset}{\value{enumi}}

The annual total assets in mutual funds in the US can be modeled by 
\[
f(x) = 126.67(1.217)^x  \quad [1, 16]
\]
where $x$ represents the number of years since 1999 and $f(x)$ represents the annual total assets in mutual funds  (in billions of dollars).
\begin{enumerate}   \setcounter{enumi}{\value{pset}}
    \item Evaluate and interpret $f(9)$
    \item Find and interpret the average rate of change of the function in the interval [5, 15].
\end{enumerate} \setcounter{pset}{\value{enumi}}

Use the simple interest formula $I = prt$, to find the total interest earned in an account given each of the following.
\begin{multicols}{2}
\begin{enumerate}   \setcounter{enumi}{\value{pset}}
    \item Deposit \$3000 at 6\% interest for 5 years
    \item Deposit \$500 at 4.5\% interest for 2 years
\end{enumerate} \setcounter{pset}{\value{enumi}}
\end{multicols}

If Mrs. Johanson deposits \$5000 into an account that yields 6.5\% annual interest, how much will be in the account after 10 years if the interest is compounded
\begin{multicols}{3}
\begin{enumerate}   \setcounter{enumi}{\value{pset}}
    \item Annually
    \item Monthly 
    \item Continuously
\end{enumerate} \setcounter{pset}{\value{enumi}}
\end{multicols}

Under certain conditions, the spread of E. Coli can be modeled by
\[
f(t) = 1,200,000e^{0.23t}
\]
where $t$ represents time in minutes and $f(t)$ represents the size of the bacteria colony.
\begin{enumerate}   \setcounter{enumi}{\value{pset}}
    \item Evaluate and interpret $f(0)$
    \item Evaluate and interpret $f(5)$
    \item Find and interpret the average rate of change in the interval [0,10].
\end{enumerate} \setcounter{pset}{\value{enumi}}

\newpage

\texttt{Key} \newline 

\begin{enumerate}
    \item 8.64
    \item 4.910
    \item 17.274
    
    \item $f(7) \approx 41.94$; after 7 days, the wound will be about 41.94 square cm.
    \item After about 3.11 days
    \item About $-19.68$; on average, the wound decreased by about 19.68 square cm each day between days 1 and 7.
    
    \item $f(9) \approx 741.80$; in 2008, the annual total assets in mutual funds was about \$741.80 billion dollars.
    \item About 207.19; on average, the total assets increased by about \$207.19 billion each year between 2004 and 2014.
    
    \item \$900
    \item \$45
    
    \item \$9,385.69
    \item \$9,560.92
    \item \$9,577.70
    
    \item $f(0) = 1,200,000$; initially, there are 1,200,000 bacteria present
    \item $f(5) \approx 3,789,831.5$; after 5 minutes, there are about 3.79 million bacteria
    \item About 1,076,901.9; in the first 10 minutes, the bacteria count grows by about 1,076,902 per minute.
\end{enumerate}




\end{document}