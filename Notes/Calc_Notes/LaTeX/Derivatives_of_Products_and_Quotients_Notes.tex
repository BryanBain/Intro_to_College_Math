\documentclass{article}
\usepackage{amsmath, tikz, tcolorbox, array, multicol, sfmath, enumerate, pgfplots}
\renewcommand{\familydefault}{\sfdefault}
\pgfplotsset{compat=newest}
\usetikzlibrary{arrows.meta}
\everymath{\displaystyle}
\tikzset{>=stealth}
\usepackage[top = 0.25in, bottom = 0.25in, left = 1.25in, right = 1.25in]{geometry}
\pagestyle{empty}
\raggedright

\newcounter{example}[section]
\newenvironment{example}[1][]{\refstepcounter{example}\par\medskip
   {\color{red}\textbf{Example~\theexample. #1}}}{\medskip}

\begin{document}

\section*{Derivatives of Products and Quotients}

\begin{tcolorbox}[colframe=orange!70!white, coltitle=black, title=\textbf{Summary}]
\begin{enumerate}
    \item 
\end{enumerate}
\end{tcolorbox}
\vspace{1in}

\subsection*{Product Rule}

Given $f(x) \cdot g(x)$, the derivative is $f'(x)\cdot g(x) + f(x) \cdot g'(x)$.

\begin{example}
Find the derivative of each.
\begin{enumerate}[(a)]
    \item $3x^3(x^4+2)$
    \item $(2x^2+4x+5)(5x-4)$
    \item $\sqrt{x}\left(3x^3-4x^2+8x\right)$
\end{enumerate}
\end{example}

\begin{example}
Extensive market research has determined that for the next 5 years the price of a certain mountain bike is predicted to vary according to $p(t) = 300 - 30x + 7.5t^2$, where $t$ is time in years and $p(t)$ is the price in dollars. \newline\\

The number of mountain bikes sold each year is expected to follow $q(t) = 3000 + 90t - 15t^2$, where $q(t)$ is the number sold and $t$ is time in years.
\begin{enumerate}[(a)]
    \item Determine $R(t)$ and $R'(t)$
    \item Compute and interpret $R'(1)$
    \item Compute and interpret $R'(4)$
\end{enumerate}
\end{example}

\subsection*{Quotient Rule}

Given $\frac{f(x)}{g(x)}$, the derivative is $\frac{f'(x)\cdot g(x) - f(x) \cdot g'(x)}{\left(g(x)\right)^2}$, where $g(x) \neq 0$.

\begin{example}
Find $\frac{dy}{dx}$ for each.
\begin{enumerate}[(a)]
    \item $y = \frac{x+3}{x-2}$
    \item $y = \frac{x^4-3x}{x^2+1}$
    \item $y = \frac{5\sqrt{x}-6}{x+1}$
\end{enumerate}
\end{example}

\begin{example}
Researchers have determined through experimentation that the percent concentration of a certain medication can be approximated by
\[
p(t) = \frac{200t}{2t^+5} - 4   \quad [0.25,20]
\]
where $t$ is the time in hours after administering the medication and $p(t)$ is the percent concentration. 
\begin{enumerate}[(a)]
    \item Evaluate and interpret $p'(1)$
    \item Evaluate and interpret $p'(6)$
\end{enumerate}
\end{example}

\end{document}
