\documentclass{article}

\usepackage{amsmath, enumerate, xcolor, tikz, pgfplots, array, bm, sfmath, tcolorbox, multicol}
\renewcommand{\familydefault}{\sfdefault}
\usepackage[top=0.5in, bottom=0.5in, left=1.25in, right=1.25in]{geometry}
\pagestyle{empty}
\raggedright

\pgfplotsset{compat = newest}
\usetikzlibrary{arrows.meta, calc, decorations.pathreplacing}
\pgfplotsset{every axis/.append style = {axis lines = middle}}
\pgfplotsset{every tick label/.append style={font=\scriptsize}}
\everymath{\displaystyle}

\newcounter{example}[section]
\newenvironment{example}[1][]{\refstepcounter{example}\par\medskip
   {\color{red}\textbf{Example~\theexample. #1}}}{\medskip}

\begin{document}

\section*{Logarithmic Functions}

\begin{tcolorbox}[colframe=orange!70!white, coltitle=black, title=\textbf{Summary}]
\begin{enumerate}
    \item The inverse of the exponential function $y = b^x$ is $y = \log_b(x)$ with $b > 0$ and $b \neq 1$
    \item $\log_{10}(x) = \log (x)$ and $\log_e (x) = \ln x$
    \item The domain of the logarithmic function is $(0, \infty)$
\end{enumerate}
\end{tcolorbox}

\vfill

Recall that an {\color{violet}\textbf{inverse function}} allows you to get your input back from your output of a function. \\[0.5in]

The \textbf{inverse} of the exponential function $f(x) = b^x$  is the {\color{blue}\textbf{logarithmic function}}. \newline\\ 

It is denoted $f^{-1}(x) = \log_b x$

\begin{align*}
    \log_b(y) &= x \\[24pt]
    b^x &= y
\end{align*}

\vfill

\begin{tcolorbox}[colframe=green!60!black, title=\textbf{Common Logarithm}]
The \textbf{common logarithm} of a real number $x$ is $\log_{10}x$ and is usually written \[\log x\]
\end{tcolorbox}

\vfill

\begin{tcolorbox}[colframe=green!60!black, title=\textbf{Natural Logarithm}]
The \textbf{natural logarithm} of a real number $x$ is $\log_e x$ and is usually written \[\ln x\]
\end{tcolorbox}

\vfill

\newpage

\begin{example}
Re-write each as a  logarithmic expression.
\begin{multicols}{3}
\begin{enumerate}[(a)]
    \item $2^5 = 32$    
    \item $3^4 = 81$    
    \item $\left(\frac{1}{4}\right)^2 = \frac{1}{16}$  
\end{enumerate}
\end{multicols}
\vspace{1.5in}
\begin{multicols}{2}
\begin{enumerate}[(a)]  \setcounter{enumi}{3}
    \item $7^{-2} = \frac{1}{49}$   
    \item $e^x = 5$ 
\end{enumerate}
\end{multicols}
\end{example}
\vspace{1.25in}

\subsection*{Properties of Logarithms}

Exponents and logarithms have similar properties where $x > 0$ and $y > 0$  \newline\\  
\begin{center}
\setlength{\extrarowheight}{7pt}
\begin{tabular}{c|c|c}
    \textbf{Property}   &   \textbf{Exponents}                  &   \textbf{Logarithms} \\ \hline
    Product             &   $b^x \cdot b^y = b^{x+y}$           &   $\log_b (xy) = \log_b (x) + \log_b (y)$ \\[0.5cm]
    Quotient            &   $\frac{b^x}{b^y} = b^{x-y}$         & $\log_b\left(\frac{x}{y}\right) = \log_b (x) - \log_b(y)$ \\[0.5cm]
    Power               & $\left(b^x\right)^y = b^{xy}$         & $\log_b(x)^y = y\cdot \log_b(x)$ \\[0.5cm]
    \hline
    Equality            &   $b^x = b^y \Longleftrightarrow x=y$ & $\log_b(x) = \log_b(y)\Longleftrightarrow x=y$   \\[6pt]
    Zero Power  &   $b^0 = 1$   & $\log_b(1) = 0$   \\[0.5cm]
    1st Power   &   $b^1 = b$   &   $\log_b(b) = 1$ \\
\end{tabular}
\end{center}
\vspace{0.5in}

\begin{example}
Use the properties of logarithms to rewrite each of the following.
\begin{multicols}{3}
\begin{enumerate}[(a)]
    \item $\log(6 \cdot 17)$
    \item $\log_5\left(\frac{10}{3}\right)$
    \item $\log_6\left(\sqrt{11}\right)$
\end{enumerate}
\end{multicols}
\end{example}

\newpage 

\subsection*{Solving Exponential Equations}

Solving an exponential equation involves 
\begin{itemize}
    \item Isolating the exponential base
    \item Taking the logarithm of both sides
\end{itemize}
\vspace{1in}

\begin{example}
\$20,000 is deposited and compounded continuously $(A = Pe^{rt})$ at a rate of 6.5\%. 
\begin{enumerate}[(a)]  
    \item How many years until there is \$25,000 in the account?    \vspace{4in}
    \item How many years until the account value doubles?
\end{enumerate}
\end{example}

% Recall that the inverse of a function is a reflection of the graph of the function across the line $y = x$.
% \begin{tabular}{p{0.45\textwidth}p{0.45\textwidth}}
% \begin{tikzpicture}
% \begin{axis}[
% xmin = -10, xmax = 12,
% ymin = -10, ymax = 10,
% ticks = none, 
% legend pos = south east,
% legend style={draw=none}
% ]
% \addplot[color=blue, ultra thick, samples=200, smooth, domain=0.01:10] {ln(x)};
% \addlegendentry{$\log_b x,\, b > 1$}
% \addplot[color=red, thick, samples=200, smooth] {e^x};
% \addlegendentry{$b^x, \, b > 1$}
% \addplot[color=violet, dashed, domain=-9:9] {x};
% \addlegendentry{$y=x$}
% \end{axis}
% \end{tikzpicture}
% &
% \begin{tikzpicture}
% \begin{axis}[
% xmin = -10, xmax = 12,
% ymin = -10, ymax = 10,
% ticks = none, 
% legend pos = south east,
% legend style={draw=none}
% ]
% \addplot[color=blue, ultra thick, samples=200, smooth, domain=0.01:10] {ln(x)/ln(0.5)};
% \addlegendentry{$\log_b x,\, 0< b < 1$}
% \addplot[color=red, thick, samples=200, smooth] {0.5^x};
% \addlegendentry{$b^x, \, 0 < b < 1$}
% \addplot[color=violet, dashed, domain=-9:9] {x};
% \addlegendentry{$y=x$}
% \end{axis}
% \end{tikzpicture}
% \end{tabular}



% \newpage

% \subsection*{Properties of All Logarithmic Functions}
% \begin{itemize}
%     \item The domain of $\log_b x$ is $(0, \infty)$ and the range is $(-\infty, \infty)$    
%     \item $(1,0)$ is on the graph of $\log_b x$ and $x=0$ is a vertical asymptote.  
%     \item $\log_b x$ is one-to-one (has an inverse, passes HLT), continuous, and smooth.
%     \item $b^a = c$ if and only if $\log_b c = a$.   
%         \begin{itemize}
%             \item In other words, $\log_b c$ is the \textbf{exponent} you put on $b$ to get $c$. 
%         \end{itemize}
%     \item  $\log_b b^x = x$ for all $x$ and $b^{\log_b x} = x$ for all $x >0$
% \end{itemize}

% \vfill

% \subsection*{$\mathbf{f(x) = \log_b(x)}$ when $\mathbf{b > 1}$}  
% \begin{itemize}
%     \item  $f$ is always increasing.  
%     \item  $\lim_{x \to 0^+} f(x) = -\infty$ and $\lim_{x \to \infty} f(x) = \infty$  
% \end{itemize}

% \vfill
% \subsection*{$\mathbf{f(x) = \log_b(x) \text{ when } 0 < b < 1}$}
% \begin{itemize}
%     \item $f$ is always decreasing.
%     \item $\lim_{x \to 0^+} f(x) = \infty$ and $\lim_{x \to \infty} f(x) = -\infty$ 
% \end{itemize}

% \vfill

% \newpage

% \begin{example} 
% Simplify each of the following.
% \begin{enumerate}[(a)]
% \item $\log_3 81$   \vfill
% \item $\log_2 \left(\frac{1}{8}\right)$ \vfill
% \item $\log_{\sqrt{5}} 25$  \vfill
% \end{enumerate}

% \newpage

% \begin{enumerate}[(a)] \setcounter{enumi}{3}
% \item $\ln \left(\sqrt[3]{e^2}\right)$      \vfill
% \item $\log 0.001$      \vfill
% \item $2^{\log_2 8}$    \vfill
% \item $117^{-\log_{117} 6}$ \vfill
% \end{enumerate}
% \end{example}

% \newpage

% \subsection*{Domain of Logarithmic Functions}

% Up until now, the only domain restrictions we have had have been   
% \begin{itemize}
%     \item  Denominator can't = 0
%     \item  Can't take even root of a negative number
% \end{itemize}
% \vspace{0.5in}

% For logarithms: 

% \[\log_b \, (\color{blue}\textbf{positive value}) \]
% \vspace{1.5in}

% \begin{example} 
% Find the domain of each. Write your answer in interval notation.    
% \begin{enumerate}[(a)]
% \item $f(x) = 2\log(3-x) - 1$   \vfill
% \end{enumerate}

% \newpage

% \begin{enumerate}[(a)] \setcounter{enumi}{1}
% \item $g(x) = -\tfrac{2}{3}\log_8\left(x^2 + 6x - 7\right)$   \vfill
% \item $h(x) = \ln\left(\frac{x}{x-1}\right)$    \vfill 
% \end{enumerate}
% \end{example}
\end{document}
