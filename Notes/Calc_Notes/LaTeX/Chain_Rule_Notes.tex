\documentclass{article}
\usepackage{amsmath, tikz, tcolorbox, array, multicol, sfmath, enumerate, pgfplots}
\renewcommand{\familydefault}{\sfdefault}
\pgfplotsset{compat=newest}
\usetikzlibrary{arrows.meta}
\everymath{\displaystyle}
\tikzset{>=stealth}
\usepackage[top = 0.25in, bottom = 0.25in, left = 1.25in, right = 1.25in]{geometry}
\pagestyle{empty}
\raggedright

\newcounter{example}[section]
\newenvironment{example}[1][]{\refstepcounter{example}\par\medskip
   {\color{red}\textbf{Example~\theexample. #1}}}{\medskip}

\begin{document}

\section*{The Chain Rule}

\begin{tcolorbox}[colframe=orange!70!white, coltitle=black, title=\textbf{Summary}]
\begin{enumerate}
    \item 
\end{enumerate}
\end{tcolorbox}
\vspace{1in}

The chain rule is used when taking the derivative of a {\color{blue}\textbf{composition of functions.}}    \\

If $h(x) = f(g(x))$, then 
\[
h'(x) = f'(g(x)) \cdot g'(x)
\]

\begin{example}
Find the derivative of each.
\begin{multicols}{3}
\begin{enumerate}[(a)]
    \item $f(x) = (x-5)^3$
    \item $f(x) = \left(5x^3 + 3x\right)^4$
    \item $y = \left(x^2+1\right)^{15}$
\end{enumerate}
\end{multicols}
\end{example}

We can use the chain rule to find derivatives of radical and rational functions.

\begin{example}
Find the derivative of each.
\begin{enumerate}[(a)]
    \item $f(x) = \sqrt[3]{2x-4}$
    \item $y = \frac{5}{(2x-3)^2}$
\end{enumerate}
\end{example}

\subsection*{Applications}

\begin{example}
An actuary has determined that the number of people surviving over the duration of a century can be modeled by
\[
f(x) = 400\sqrt{100-x} \quad [0, 100]
\]
where $x$ represents the age of the person in years and $f(x)$ represents the number of people surviving. Evaluate and interpret $f'(70)$.
\end{example}

\begin{example}
A company has assumed that the price-demand function for their spark plug is
\[
p(x) = \frac{125}{\sqrt{2x+5}} \quad [0, 20]
\]
where $x$ represents the number of spark plugs manufactured in hundreds and $p(x)$ is the price of the spark plug.
\begin{enumerate}[(a)]
    \item Determine and interpret $p'(20)$
    \item Determine the revenue function $R(x)$
    \item Determine and interpret $R'(20)$
\end{enumerate}
\end{example}

\end{document}
