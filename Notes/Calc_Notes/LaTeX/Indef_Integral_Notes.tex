\documentclass{article}
\usepackage{amsmath, tikz, tcolorbox, array, multicol, sfmath, enumerate, pgfplots}
\renewcommand{\familydefault}{\sfdefault}
\pgfplotsset{compat=newest}
\usetikzlibrary{arrows.meta}
\everymath{\displaystyle}
\tikzset{>=stealth}
\usepackage[top = 0.25in, bottom = 0.25in, left = 1.25in, right = 1.25in]{geometry}
\pagestyle{empty}
\raggedright

\newcounter{example}[section]
\newenvironment{example}[1][]{\refstepcounter{example}\par\medskip
   {\color{red}\textbf{Example~\theexample. #1}}}{\medskip}
\newcommand{\dx}{\, \mathrm{d}x}

\begin{document}

\section*{Indefinite Integrals}

\begin{tcolorbox}[colframe=orange!70!white, coltitle=black, title=\textbf{Summary}]
\begin{enumerate}
    \item 
\end{enumerate}
\end{tcolorbox}
\vspace{1in}

The WorkShop Company determines the marginal cost function for their designer suspenders is
\[
MC(x) = C'(x) = 2x  \quad [0, 10]
\]
where $x$ is the number of designer suspenders produced in thousands and $MC(x)$ is the marginal cost in thousands of dollars.   \newline\\

\begin{itemize}
    \item Given this information, how can we find the cost function?
    \item We need to find a function $C(x)$ such that
\[
C'(x) = MC(x)
\]
    \item What function, when we take its derivative, gives us $2x$?
    \item That function is the {\color{blue}\textbf{antiderivative}} of the function.
        \subitem Antidifferentiation is the inverse of differentiation.
\end{itemize}

\begin{center}  \setlength{\extrarowheight}{6pt}
    \begin{tabular}{ccc}    
        Function & Derivative & Rule \\ \hline 
        $f(x)=x^n$ & $f'(x)=n\cdot x^{n-1}$ & Power Rule \\
        $f(x) = k$ & $f'(x) = 0$ & Constant Rule \\
        $h(x) = f(x) \pm g(x)$ & $h'(x) = f'(x) \pm g'(x)$ & Sum and Difference Rules
    \end{tabular}
\end{center}

What function has a derivative of $2x$? \\

Other functions with derivative of $2x$:
\begin{itemize}
    \item $x^2 + 3$
    \item $x^2 - 1.7$
    \item $x^2 + \tfrac{1}{5}$
\end{itemize}

If $C$ is any real number, then $x^2 + C$ is the antiderivative of $2x$. \\

$C$ is called an {\color{blue}\textbf{arbitrary constant}}.

\begin{example}
Determine if the function $F$ is the general antiderivative of the function $f$.
\begin{enumerate}[(a)]
    \item $F(x) = \tfrac{2}{3}x^{3/2} + 4x + C; \, f(x) = \sqrt{x}+4$
    \item $f(x) = 2x^4 - x + C; \, f(x) = \tfrac{2}{3}x^3-1$
\end{enumerate}
\end{example}

Another way to represent the general antiderivative of a function $f$ is by the {\color{blue}\textbf{indefinite integral}}
\[
\int f(x) \dx
\]

Note that, from Example 1a, $\int \left(\sqrt{x}+4\right) \dx = \tfrac{2}{3}x^{3/2}+4x+C$ \\

\texttt{Power Rule for Integration}

For $n \neq -1$,
\[
\int x^n \dx = \frac{1}{n+1}x^{n+1} + C
\]

\begin{example}
Determine the following indefinite integrals.
\begin{enumerate}[(a)]
    \item $\int x^8 \dx$
    \item $\int \sqrt[4]{x} \dx$
    \item $\int \frac{1}{x^5} \dx$
\end{enumerate}
\end{example}

\texttt{Constant Rule for Integration}

If $k$ is a real number, then 
\[
\int k \dx = kx + C
\]

\texttt{Sum and Difference Rules for Integration}
\[
\int \left(f(x) \pm g(x)\right) \dx = \int f(x) \dx \pm \int g(x) \dx
\]

\begin{example}
Determine each indefinite integral.
\begin{enumerate}[(a)]
    \item $\int \left(x^2 + 3\right) \dx$
    \item $\int \left(\sqrt[3]{x} + 5\right) \dx$
\end{enumerate}
\end{example}

\texttt{Coefficient Rule for Integration}
\[
\int c \cdot f(x) \dx = c \cdot \int f(x) \dx
\]

\begin{example}
Determine $\int \left(-2t^3 + 3t + 5\right) \, \mathrm{d}t$
\end{example}

\end{document}
