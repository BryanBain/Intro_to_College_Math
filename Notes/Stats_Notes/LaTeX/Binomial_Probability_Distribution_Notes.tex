\documentclass{article}
\usepackage{amsmath, tikz, tcolorbox, array, multicol, sfmath, enumerate, pgfplots}
\renewcommand{\familydefault}{\sfdefault}
\pgfplotsset{compat=newest}
\usepackage[margin = 0.5in]{geometry}
\pagestyle{empty}
\raggedright

\newcounter{example}[section]
\newenvironment{example}[1][]{\refstepcounter{example}\par\medskip
   {\color{red}\textbf{Example~\theexample. #1}}}{\medskip}
\newcommand{\dx}{\, \mathrm{d}x}

\begin{document}

\section*{Binomial Probability Distribution}

\begin{tcolorbox}[colframe=orange!70!white, coltitle=black, title=\textbf{Summary}]
\begin{enumerate}
    \item Binomial distributions are a special type of probability distribution that allows us to find outcomes that are relevant to two categories: success or failure.
    \item The formula for calculating binomial probability distribution is 
    \[	P(x) = {_n}C_r \cdot p^x \cdot q^{n-x}	\]
for $x = 0, \, 1, \, 2, \, \dots, n$
\end{enumerate}
\end{tcolorbox}
\vspace{0.25in}

\subsection*{Qualifications To Be a Binomial Probability Distribution}

\begin{enumerate}
	\item The outcomes are success or failure
	\item There is a fixed number of trials $n$
	\item The probability of success, $p$, never changes
	\item The outcomes are independent of one another
	\end{enumerate}
\vspace{0.25in}
	
\begin{example}
Decide whether each of the following experiments is a binomial experiment. If it is, specify the values for $n$, $p$, and $x$. If it isn't, explain why.
	\begin{enumerate}[(a)]  \setlength{\itemsep}{0.75in}
	\item A certain surgical procedure has an 85\% success rate. A doctor performs the procedure on 8 patients. The random variable represents the number of successful surgeries for the 8 patients.
	\item A jar contains 5 red marbles, 9 blue marbles, and 4 green marbles. You randomly select three marbles without replacement. The random variable represents the number of red marbles.
	\end{enumerate}
\end{example}
\vspace{0.5in}

\subsection*{Developing the Formula}

We can list out one possible example for a binomial distribution as follows: \textit{SFS}. Note that there are 2 successes and 1 failure. This is also the same result as \textit{FSS} and \textit{SSF}.
\newline\\


Thus, \underline{the order in which success and failure occur in a binomial distribution is not important}. Hence, combinations will play a factor in the binomial formula.
\newline\\


Also, if we let $q$ represent the probability of failure, then the probability of $x$ successes out of $n$ trials is calculated by the Binomial Formula:
	\[	P(x) = {_n}C_r \cdot p^x \cdot q^{n-x}	\]
for $x = 0, \, 1, \, 2, \, \dots, n$.

\newpage 

\begin{example}
A certain surgical procedure has an 85\% success rate. A doctor performs the procedure on 8 patients. Find the probability that the surgery is successful for
\begin{multicols}{3}
	\begin{enumerate}[(a)]
	\item 5 patients
	\item 8 patients
	\item 0 patients
	\end{enumerate}
\end{multicols}
\end{example}

\vspace{1in}

We can also use cumulative binomial functions to find the results of adding from 0 to $x$ number of successes.

\begin{example}
A certain surgical procedure has an 85\% success rate. A doctor performs the procedure on 8 patients. Find the probability that the surgery is successful for
\begin{multicols}{3}
	\begin{enumerate}[(a)]
	\item 6 patients or less
	\item At least 3 patients
	\item Fewer than 5 patients
	\end{enumerate}
\end{multicols}
\end{example}

\vspace{1in}

\subsection*{Parameters for Binomial Distributions} 
\vspace{0.15in}

\begin{center}
\begin{tabular}{cc}
    \textbf{Mean (Expected Value):} & $\mu = np$ \\
    \textbf{Variance:} & $\sigma^2 = npq$ \\
    \textbf{Standard Deviation:} & $\sigma = \sqrt{npq}$
\end{tabular}
\end{center}

\begin{example}
\begin{enumerate}[(a)]  \setlength{\itemsep}{0.75in}
    \item According to recent numbers, only 4\% of people have type AB blood. Out of a group of 500 people, how many of them would you \emph{expect} to have type AB blood?
    \item What is the standard deviation for the number of people with type AB blood?
    \item Recall that the range rule of thumb can be used to find minimum and maximum ``usual" values.
\begin{center}
Min Usual Value = $\mu - 2\sigma$	\\
Max Usual Value = $\mu + 2\sigma$
\end{center}

For groups of 500 randomly selected people, find the minimum and maximum usual number of people with type AB blood.
\end{enumerate}
\end{example}

\newpage 

\begin{example}
\begin{enumerate}[(a)]  \setlength{\itemsep}{0.75in}
    \item You take a multiple-choice test that consists of 25 questions. If each answer has 4 choices (A--D), what is the expected number of questions you would get correct by pure guessing?
    \item What is the standard deviation for the number of questions guessed correctly?
    \item Find the minimum and maximum ``usual`` number of questions guessed correctly. Does this seem like a good strategy for test preparation?
\end{enumerate}
\end{example}


\end{document}
