\documentclass{article}
\usepackage[top=0.5in, bottom=0.5in, left=1.25in, right=1.25in]{geometry}

\usepackage{amsmath, array, enumerate, tikz, bm, pgfplots, pgfplotstable, tcolorbox, graphicx, venndiagram, color, colortbl}
\pgfplotsset{compat = newest}
\usepgfplotslibrary{statistics}
\renewcommand{\familydefault}{\sfdefault}
\raggedright
\pagestyle{empty}

\newcounter{example}[section]
\newenvironment{example}[1][]{\refstepcounter{example}\par\medskip
   {\color{red}\textbf{Example~\theexample. #1}}}{\medskip}

\begin{document}

\section*{Measures of Center}

\begin{tcolorbox}[colframe=orange!70!white, coltitle=black, title=\textbf{Summary}]
\begin{enumerate}
    \item Measures of center give us a ``starting point" for our data set.
\end{enumerate}
\end{tcolorbox}
\vspace{0.75in}


\subsection*{The Mean}
\begin{tcolorbox}[colframe=green!20!black, colback = green!30!white,title=\textbf{Mean}]
The \textbf{mean} of a data set is found by adding all of the data values together and then dividing by the total number of values. 
\end{tcolorbox}
\vspace{11pt}	

When most people use the term \textit{average}, they are referring to the mean.


\subsubsection*{Properties of the Mean}
\begin{itemize}
	\item Sample means drawn from the same population tend to vary less than other measures of center.	
	\item The mean of a data set uses every value, unless the mean is a \textit{trimmed mean}.	
	\item One extreme value (called an {\color{blue}\textbf{outlier}}) can change the value of the mean drastically.
\end{itemize}
\vspace{0.5in}

\subsubsection*{Mean Formula}
Sample mean: $\overline{x}$	\qquad	Population mean: $\mu$	\newline\\	
\[ \overline{x}, \text{ or } \mu, = \frac{\sum x_i}{n} = \frac{1}{n}\sum x_i	\]

\vfill 


\subsection*{The Median}

\begin{tcolorbox}[colframe=green!20!black, colback = green!30!white,title=\textbf{Median}]
The \textbf{median} of a data set is found by first arranging the data values from least to greatest, then selecting the data value in the middle. 
\end{tcolorbox}
\vspace{0.5in}


\subsubsection*{Properties of the Median}
\begin{itemize}
    \item Denoted by $\tilde{x}$ or \texttt{Med}.
    \item If there are 2 data values in the middle, the median will be the mean of these two values.
    \item Separates the top 50\% of the data from the bottom 50\%.
    \item Typically does not change by large amounts when including extreme values (median is \textbf{resistant})
\end{itemize}

\newpage 


\subsection*{The Mode}

\begin{tcolorbox}[colframe=green!20!black, colback = green!30!white,title=\textbf{Mode}]
The \textbf{mode} of a data set is the value(s) that occur the most.
\end{tcolorbox}
\vspace{0.25in}

\begin{itemize}
    \item May be one mode, no mode, or many modes (2 modes is called \textbf{bimodal}).
    \item Only measure of center to use on qualitative data.
\end{itemize}

\vspace{0.75in}

\begin{example}
The data set below represents the number of complaints I receive each week about my teaching.
\begin{center}
8, 2, 3, 7, 4, 4, 1, 9, 7, 5
\end{center}
\vspace{8pt}

Calculate the mean, median, and mode of the number of complaints.
\end{example}
\vfill 

\begin{example}
The next week, I received 400 complaints. Re-calculate the mean, median, and mode.
\end{example}
\vfill 

\begin{example}
California has a mean class size of 20.9 students per teacher and Alaska has a mean of 16.8 students per teacher. \newline\\ 

Combining the two states, the mean number of students per teacher to be 18.85, $\left(\frac{20.9+16.8}{2}\right)$, but is this result correct? Why or why not?
\end{example}

\vfill
\newpage

\subsection*{Weighted Mean}
Sometimes it is necessary to take into account how large each class of a data set is.
\vspace{0.25in}

If $w_i$ represents the {\color{blue}\textbf{weight}} of each class, then the \textbf{weighted mean} can be found via

\[\frac{\sum \left(x_i \cdot w_i\right)}{\sum w_i}\]
\vspace{0.25in}

\begin{center}
\fbox{\parbox{3.5in}{
In other words: 
\begin{enumerate}
	\item Multiply each data value by its corresponding weight.	
	\item Add those results.
	\item Then divide that by the total of the weights.
\end{enumerate}
}}
\end{center}
\vspace{1in}

\begin{example}
You've recently completed a semester in college. Determine the semester's GPA (A = 4pts, B = 3pts, etc).	\newline\\
\begin{center}
\begin{tabular}{|c|c|c|}
\textbf{Course} 			&	\textbf{Grade}	&	\textbf{Credit Hours} 	\\	\hline
Statistics					&	A 				&	4						\\
Advanced Chris Farley		&	A 				&	3						\\
\textit{Airplane!} Quotes	&	B 				&	5						\\
Obnoxious Examples			&	C 				&	3						\\
\end{tabular}
\end{center}
\end{example}

\vfill 

\begin{example}
In a statistics course, tests count for 60\% of the final grade, homework for 20\% and midterm and final exams are 10\% each. Suppose you've earned an 87\% average on tests, 94\% average on homework and a 77\% average on the exams. \newline\\ 
What is your overall percentage for the course?	
\end{example}

\vfill 
\newpage 

\subsection*{Grouped Data}

\begin{itemize}
    \item In grouped form, we don't know individual values in our data set.
    \item Mean can just be an educated guess.
    \item Use class midpoints and weighted mean techniques.
\end{itemize}
\vspace{1in}

\begin{example}
The table below gives the number of sushi rolls various tables ordered at a Japanese restaurant one day.
\begin{center}
\begin{tabular}{|c|c|}
\textbf{Number of Rolls} & \textbf{Frequency} \\ \hline
1 -- 5		&	4	\\
6 -- 10		&	17	\\
11 -- 15	&	12	\\
\end{tabular}
\end{center} 
\vspace{15pt}

Estimate the mean number of sushi rolls consumed per table.
\end{example}

\end{document}
