\documentclass{article}
\usepackage{amsmath, tikz, tcolorbox, array, multicol, sfmath, enumerate, pgfplots}
\renewcommand{\familydefault}{\sfdefault}
\pgfplotsset{compat=newest}
\usepackage[top = 0.5in, right = 1.25in, bottom =0.5in, left = 1.25in]{geometry}
\pagestyle{empty}
\raggedright

\newcounter{example}[section]
\newenvironment{example}[1][]{\refstepcounter{example}\par\medskip
   {\color{red}\textbf{Example~\theexample. #1}}}{\medskip}

\begin{document}

\section*{Normal Probability Distribution}

\begin{tcolorbox}[colframe=orange!70!white, coltitle=black, title=\textbf{Summary}]
\begin{enumerate}
    \item The normal probability distribution is a common probability distribution often called a ``bell curve."
    \item Normal probability distribution is a very common one in statistics.
\end{enumerate}
\end{tcolorbox}
\vspace{0.25in}

\subsection*{The Standard Normal Distribution}

A {\color{blue}\textbf{standard normal distribution}} is a continuous probability distribution, utilizing z-scores, with the following properties:
\begin{itemize}
	\item The graph is bell-shaped
	\item The mean is 0
	\item The standard deviation is 1
    \item The total area under the graph is equal to 1.
	\item There is a correspondence between area and probability (relative frequency). Some probabilities can be found by identifying the corresponding areas in the graph.
	\begin{itemize}
	    \item For instance, the area to the left of $z = 0.24$ is denoted $P(z < 0.24)$
	    \item The area to the right of $z = 2.12$ is $P(z > 2.12)$
	    \item The area between $z = -1.41$ and $z = 1.55$ is $P(-1.41 < z < 1.55)$
	    \item The area of any particular $z$-score is 0; \textit{i.e.} $P(z = 1)$ is 0
	\end{itemize}
	\item The curve cannot fall below the $x$-axis.
\end{itemize}

\vspace{0.5in}

\subsubsection*{Find Area/Probability Given Z-Score(s)}

\vspace{0.5in}

\begin{center}
\fbox{\parbox{3.75in}{
\texttt{Pro Tip:} \newline

Sketch a curve when finding probabilities of normal distributions.}}
\end{center}

\vspace{0.5in}

\begin{example}
Find each probability/area.
\begin{multicols}{3}
\begin{enumerate}[(a)]
	\item $P(z < 0.24)$
	\item $P(z > 2.12)$
	\item $P(-1.41 < z < 1.55)$
\end{enumerate}
\end{multicols}
\end{example}

\newpage 

\subsubsection*{Find Z-Score For a Given Area/Probability}

Area is \emph{cumulative} and will typically be coming in from the \textbf{\underline{left}} on the graph. \newline\\

Still a good idea to sketch the graphs.

\vspace{0.5in}

\begin{example}
Find the $z$-score that corresponds to an area of 0.4216 to the left. Round your answer to 2 decimal places.
\end{example}

\vfill 

\begin{example}
Find the $z$-score that corresponds to the 90th percentile.
\end{example}

\vfill 

\begin{example}
Find the $z$-scores that correspond to each.
\begin{multicols}{3}
\begin{enumerate}[(a)]
    \item Middle 90\%
    \item Middle 95\%
    \item Middle 99\%
\end{enumerate}
\end{multicols}
\end{example}

\vfill 
\newpage 

\subsection*{Applied Normal Probability Distribution}

\begin{itemize}
    \item We could find $z$-score of an observed value if we know mean and standard deviation.
    \item Technology allows us to avoid this process.
\end{itemize}

\begin{example}
A classic example of normal probability distribution is IQ scores. Most IQ tests have a mean of 100 and a standard deviation of 15. Find each probability.
\begin{enumerate}[(a)]
\begin{multicols}{3}
    \item $P(\text{IQ} > 100)$
    \item $P(\text{IQ} \geq 100)$
    \item $P(\text{IQ} < 110)$
\end{multicols}
\vfill 
\begin{multicols}{3}
    \item $P(95 < \text{IQ} < 125)$
    \item $P(\text{IQ} > 135)$
    \item $P(\text{IQ} < 80)$
\end{multicols}
\end{enumerate}
\end{example}
\vfill 
\begin{quote}
``People who boast about their IQ are losers." \newline\\
    -- Stephen Hawking
\end{quote}

\newpage 

\subsubsection*{Find Specific Value(s) For a Given Area/Probability}

\begin{example}
A study was done to test the reaction time of subjects under poor lighting to simulate evening-time driving. \newline

Subjects were to press a buzzer as soon as they saw something appear on the side of their screen. \newline

The mean reaction time was 0.85 seconds with a standard deviation of 0.18 seconds. \newline 

Find the times corresponding to each percentile score.
\begin{enumerate}[(a)]
\begin{multicols}{3}
    \item 25th percentile
    \item 50th percentile
    \item 95th percentile
\end{multicols}
\end{enumerate}
\end{example}

\end{document}
